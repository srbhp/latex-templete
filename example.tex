\documentclass{beamer}
\usepackage{listings,microtype,textcomp,lmodern,fancybox,wasysym,tcolorbox,tikz,fontspec}
\usepackage[T1]{fontenc}
%%%%%%%%%%Modify default Color and style%%%%%%%%%%%%%%%%%%%%%%%%%%
% defaults
\useinnertheme[shadow]{rounded}
\usecolortheme{spruce}
% fonts
\usepackage{xltxtra}
\usepackage{polyglossia}
\setsansfont[BoldFont={FiraCode Nerd Font, Bold}]{FiraCode Nerd Font}
\setbeamerfont{frametitle}{size=\Large,series=\bfseries}
\setbeamerfont{title}{size=\Large,series=\bfseries}
% set frametitle shadow and rounded in the buttom corners 


\tcbuselibrary{skins, breakable}


\setbeamertemplate{frametitle}
{
    \nointerlineskip
    \begin{beamercolorbox}[sep=0.3cm,ht=1.8em,wd=\paperwidth,shadow=true,rounded = true]{frametitle}
        \vbox{}\vskip-2ex%
        \strut\insertframetitle\strut
        \vskip-0.8ex%
    \end{beamercolorbox}
	\hfill
	%\includegraphics[width=1cm]{logo}
}




\setbeamertemplate{itemize/enumerate subbody begin}{\begin{tcolorbox}[colback=white!5!white,colframe=blue!75!black,sharp corners=south]}
\setbeamertemplate{itemize/enumerate subbody end}{\end{tcolorbox}}
% hide navigation symbols
\setbeamertemplate{navigation symbols}{}


%%%%%%%%%%%%%%%%%%%%%%%%%%%%%%%%%%%%%%%%%%%%%%%%%%%%%%%%%%%%%%%%%%
\title{Very Very long	Title of the Presentation that is so long}
\subtitle{Subtitle of the Presentation that is also very long}
\author{ author 1 \\ author 2 \\ author 3 \\ \vspace{0.5cm}}

\institute{Sample  University  \\ \vspace{1cm}
  %\includegraphics[scale=0.15]{imag2/wip.png}
}
\date{\today}

\begin{document}

\begin{frame}[plain]
	\titlepage
\end{frame}

\begin{frame}
	\frametitle{Outline}
	\begin{tcolorbox}[colback=red!5!white,colframe=red!35!black,sharp corners=northwest ]
		\textbf{more} topics that are not listed here but will be discussed in the presentation.
	\end{tcolorbox}
\end{frame}


\textsf{This is a sample text in Fira Code Nerd Font.}
\textsf{\textbf{This is bold text in Fira Code Nerd Font.}}


\section{Introduction}


\begin{frame}
	\frametitle{Band Theory}
	\begin{columns}
		\begin{column}{0.5\textwidth}
			\begin{itemize}
				\item \textbf{Band theory} : electrons in solids are not bound to individual atoms, but are shared by many atoms
				\item \textbf{Fermi energy} : highest energy level occupied at absolute zero temperature
				\item \textbf{Valence band} : band of energy levels occupied by electrons
				\item \textbf{Conduction band} : band of energy levels that are empty at absolute zero temperature
			\end{itemize}
		\end{column}
		\begin{column}{0.5\textwidth}
			\begin{itemize}
				\item Metals have lot of electrons near Fermi energy
				\item Resistivity increases with temperature \\
				      due to \textbf{defects, lattice vibration }
				\begin{itemize}
					\item \textbf{Defects} : impurities, vacancies, dislocations
					\item \textbf{Lattice vibration} : phonons
				\end{itemize}
			\end{itemize}
		\end{column}
	\end{columns}
\end{frame}

\section{Results}
\begin{frame}
	\frametitle{Resistivity of Metals }

	\begin{itemize}
		\item \textbf{ Drude theory }: metals are good electrical conductors because electrons can move nearly freely between the atoms in solids

		\item conductivity: \Large{ $\sigma \propto \tau \Rightarrow \rho \propto \tau^{-1} $ } \\

		      $\tau$ : average lifetime of the electron between collision with other \textbf{electrons, impurities, lattice}
		      \begin{align}
			      \tau^{-1} & = \tau^{-1}_{imp} + \tau^{-1}_{el-el} + \tau^{-1}_{el-ph}\nonumber \\
			      \rho      & = \rho_0 + aT^2 + bT^5 \nonumber
		      \end{align}

		      \textcolor{red}{Resistivity increases with Temperature}
	\end{itemize}
\end{frame}

\section{Theory}
\begin{frame}[plain]
	\begin{alertblock}{Future Plan}
		\begin{itemize}
			\item Strong coupling between lead and impurity
			\item Non-equilibrium effects
			\item Ferromagnetic RKKY (?)
			\item Generalize to multi-impurity
		\end{itemize}
	\end{alertblock}
\end{frame}

\section{Theory}
\begin{frame}[plain]
	\begin{block}{Future Plan}
		\begin{itemize}
			\item Strong coupling between lead and impurity
			\item Non-equilibrium effects
			\item Ferromagnetic RKKY (?)
			\item Generalize to multi-impurity
		\end{itemize}
	\end{block}
\end{frame}


\begin{frame}[plain]
	\begin{center}
		{\fontsize{50}{60}\selectfont \textbf{ Thank You.}}
	\end{center}
\end{frame}
\end{document}
